\MakeAbstract{
    论文全面探讨了深度强化学习的理论基础、当前算法及其在多个应用领域中的实际效用,特别是在自动驾驶汽车领域。文中介绍了深度强化学习的基本概念,强调它如何结合强化学习和深度学习技术处理高维感知数据和复杂决策问题。详细介绍了基于值函数和策略梯度的主要深度强化学习算法,并进一步探讨了这些算法在各个领域中的具体应用。最后,文章讨论了深度强化学习面临的挑战,包括算法稳定性、计算资源需求和实时决策速度,指出未来研究的可能方向,如提高算法效率、优化策略泛化能力以及确保人工智能系统的伦理和安全。
}{深度强化学习, 自动驾驶, 调研}

\MakeAbstractEng{
    An abstract is usually a short summary of an article, essay, report, or other text. Its purpose is to help the reader understand the main content and conclusions of the text so that he or she can decide whether he or she needs to continue reading the original text. The abstract usually contains information about the topic, purpose, methods, results, and conclusions of the text and is presented as concisely and clearly as possible. A good abstract should be able to summarize the main points of the text while avoiding unnecessary details and jargon so that it can be easily understood by a wide audience.

    In addition, abstracts are often one of the most important bases on which academics and researchers evaluate a piece of literature. During the literature search and selection process, people often base their decision to look further into the complete literature on the abstract. Therefore, writing a clear, accurate, and concise abstract is crucial to the dissemination and impact of the literature. When writing an abstract, authors should follow the formatting requirements and writing specifications of the literature, as well as combine the content and purpose of the text to write an accurate, concise, and easy-to-understand abstract in order to improve the dissemination and impact of the literature.

    Keyword 1, Keyword 2, and Keyword 3 are usually a few words related to the content of the article and are used to help readers better understand the topic and content of the article. The choice of keywords should be closely related to the topic and research area of the article, and words that are representative, authoritative, unique, and searchable should usually be chosen.
}{Keyword 1, Keyword 2, Keyword 3}
