\section{物理设计}

\subsection{字段命名规范}

在数据库设计过程中,遵循良好的字段命名规范不仅有助于提高可读性和维护性,还能减少错误并提高团队协作效率。首先,选择一种统一的命名风格并在整个数据库中保持一致是关键,如驼峰命名法(camelCase)或下划线命名法(snake\_case)。字段名称应清晰描述字段的含义和用途,避免使用模糊或缩写的名称。使用 Email 而不是 EM。此外,避免使用数据库管理系统的保留字作为字段名称,并确保名称拼写正确,防止引起歧义和误解。

在设计各个表时,还应注意字段名称的长度适中,既不能过长,也不能过于简短以至于失去意义。例如,设计用户表 (User) 时,可以使用 UserID 作为主键,Username 表示用户登录名,Password 表示用户密码,Email 表示用户的电子邮件地址,Contact 表示用户的联系电话,以及 UserType 表示用户的角色。对于外键字段,可以使用特定的后缀(如 ID)来表明其引用的是另一个表的主键,这样可以清晰地展示字段之间的关系。

\subsection{索引设计}

在数据库索引设计中,首先,我们应该为数据量大、查询频繁的表建立索引,并对常作为查询条件的字段建立索引。其次,选择区分度高的列作为索引,尽量建立唯一索引,对于字符串类型的长字段可以建立前缀索引。最后,使用联合索引以减少单列索引的数量,并在涉及索引的列上使用 NOT NULL 约束,以避免空值对索引性能的影响。

\begin{table}[!hpt]
    \caption{用户表}
    \label{tab:rela-user}
    \centering
    \begin{tabular}{@{}llll@{}} \toprule
        \textbf{属性名} & \textbf{说明} & \textbf{数据类型} & \textbf{约束} (\$) \\ \midrule
        UserID & 用户ID & INT & PRIMARY KEY \\
        Username & 用户名 & VARCHAR(50) & NOT NULL \\
        Password & 密码 & VARCHAR(50) & NOT NULL \\
        Email & 邮箱 & VARCHAR(100) & NOT NULL, UNIQUE \\
        Contact & 联系方式 & VARCHAR(15) &  \\
        UserType & 用户类型 & VARCHAR(20) & NOT NULL \\ \bottomrule
    \end{tabular}
\end{table}

\subsection{完整性约束}

\paragraph{用户表}

主键约束:UserID 是用户表的主键,确保每个用户有唯一标识。

唯一约束:Email 是唯一的,确保每个用户名和电子邮件地址在系统中是唯一的。

非空约束:Username、Password、Email 和 UserType 是非空的,确保这些关键字段不会为空。

\paragraph{事件表}

主键约束:EventID 是事件表的主键,确保每个事件有唯一标识。

非空约束:EventType、Description、EventTime、Location、ReporterID 和 Status 是非空的,确保这些关键字段不会为空。

参照完整性:ReporterID 是外键,引用 User 表的 UserID,确保事件的上报者必须是系统中的有效用户。

\paragraph{多媒体文件表}

主键约束:FileID 是多媒体文件表的主键,确保每个文件有唯一标识。

非空约束:EventID、FilePath 和 FileType 是非空的,确保这些关键字段不会为空。

参照完整性:EventID 是外键,引用 Event 表的 EventID,确保每个文件必须关联到一个有效的事件。

\paragraph{通知表}

主键约束:NotifyID 是通知表的主键,确保每个通知有唯一标识。

非空约束:ReceiverID、Content、SendTime 和 Status 是非空的,确保这些关键字段不会为空。

参照完整性:ReceiverID 是外键,引用 User 表的 UserID,确保每个通知的接收者必须是系统中的有效用户。

\paragraph{用户事件表}

主键约束:UserID 和 EventID 共同组成复合主键,确保每个用户和事件的关联关系唯一。

非空约束:UserID 和 EventID 是非空的,确保这些关键字段不会为空。

参照完整性:

UserID 是外键,引用 User 表的 UserID,确保每个记录中的用户必须是系统中的有效用户。
EventID 是外键,引用 Event 表的 EventID,确保每个记录中的事件必须是系统中的有效事件。
