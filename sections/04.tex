\section{逻辑设计}

\subsection{关系模型设计}

\paragraph{用户表}

User(\underline{UserID}, Username, Password, Email, Contact, UserType);

\paragraph{事件表}

Event(\underline{EventID}, EventType, Description, EventTime, Location, ReporterID, Status);

\paragraph{多媒体文件表}

MediaFile(\underline{FileID}, EventID, FilePath, FileType);

\paragraph{通知表}

Notification(\underline{NotifyID}, ReceiverID, Content, SendTime, Status);

\paragraph{用户事件表}

UserEvent(\underline{UserID}, \underline{EventID});

\subsection{关系表设计}

\begin{table}[!hpt]
    \caption{用户表}
    \label{tab:rela-user}
    \centering
    \begin{tabular}{@{}llll@{}} \toprule
        \textbf{属性名} & \textbf{说明} & \textbf{数据类型} & \textbf{约束} \\ \midrule
        UserID & 用户ID & INT & PRIMARY KEY \\
        Username & 用户名 & VARCHAR(50) & NOT NULL \\
        Password & 密码 & VARCHAR(50) & NOT NULL \\
        Email & 邮箱 & VARCHAR(100) & NOT NULL, UNIQUE \\
        Contact & 联系方式 & VARCHAR(15) &  \\
        UserType & 用户类型 & VARCHAR(20) & NOT NULL \\ \bottomrule
    \end{tabular}
\end{table}

\begin{table}[!hpt]
    \caption{事件表}
    \label{tab:rela-event}
    \centering
    \begin{tabular}{@{}llll@{}} \toprule
        \textbf{属性名} & \textbf{说明} & \textbf{数据类型} & \textbf{约束} \\ \midrule
        EventID & 事件ID & INT & PRIMARY KEY \\
        EventType & 事件类型 & VARCHAR(50) & NOT NULL \\
        Description & 事件描述 & VARCHAR(50) & NOT NULL \\
        EventTime & 事件发生时间 & DATETIME & NOT NULL \\
        Location & 事件发生地点 & VARCHAR(100) & NOT NULL \\
        ReporterID & 上报用户ID & INT & FOREIGN KEY REFERENCES User(UserID) \\
        Status & 审核状态 & VARCHAR(20) & NOT NULL \\ \bottomrule
    \end{tabular}
\end{table}

\begin{table}[!hpt]
    \caption{多媒体文件表}
    \label{tab:rela-mediafile}
    \centering
    \begin{tabular}{@{}llll@{}} \toprule
        \textbf{属性名} & \textbf{说明} & \textbf{数据类型} & \textbf{约束} \\ \midrule
        FileID & 文件ID & INT & PRIMARY KEY \\
        EventID & 事件ID & INT & FOREIGN KEY REFERENCES Event(EventID) \\
        FilePath & 文件路径 & VARCHAR(255) & NOT NULL \\
        FileType & 文件类型 & VARCHAR(20) & NOT NULL \\ \bottomrule
    \end{tabular}
\end{table}

\begin{table}[!hpt]
    \caption{通知表}
    \label{tab:rela-notification}
    \centering
    \begin{tabular}{@{}llll@{}} \toprule
        \textbf{属性名} & \textbf{说明} & \textbf{数据类型} & \textbf{约束} \\ \midrule
        NotifyID & 通知ID & INT & PRIMARY KEY \\
        ReceiverID & 接收用户ID & INT & FOREIGN KEY REFERENCES User(UserID) \\
        Content & 通知内容 & TEXT & NOT NULL \\
        SendTime & 发送时间 & DATETIME & NOT NULL \\
        Status & 状态 & VARCHAR(20) & NOT NULL \\ \bottomrule
    \end{tabular}
\end{table}

\begin{table}[!hpt]
    \caption{用户事件表}
    \label{tab:rela-user-event}
    \centering
    \begin{tabular}{@{}llll@{}} \toprule
        \textbf{属性名} & \textbf{说明} & \textbf{数据类型} & \textbf{约束} \\ \midrule
        UserID & 用户ID & INT & FOREIGN KEY REFERENCES User(UserID) \\
        EventID & 事件ID & INT & FOREIGN KEY REFERENCES Event(EventID) \\
         & & & PRIMARY KEY (UserID, EventID) \\ \bottomrule
    \end{tabular}
\end{table}

\subsection{3NF满足判断}

为了判断城市犯罪事件管理平台的数据库设计是否满足第三范式 (3NF),我们需要分析每个关系,并确保它们满足以下条件:

第一范式 (1NF):表中的每个列都包含不可再分的原子值,没有重复的列。

第二范式 (2NF):满足 1NF,并且表中的每列都完全依赖于候选键,而不是部分依赖。

第三范式 (3NF):满足 2NF,并且表中的每列都与候选键直接相关,而不是传递相关。

\paragraph{用户表}

列:用户 ID (主键)、用户名、密码、邮箱、联系方式、用户类型

1NF:每个列包含不可再分的原子值,没有重复的列。

2NF:每个列都完全依赖于主键 (UserID),不存在部分依赖。

3NF:每个列都直接依赖于主键 (UserID),不存在传递依赖。

结论:用户表满足 1NF、2NF 和 3NF。

\paragraph{事件表}

列:事件 ID (主键)、事件类型、事件描述、发生时间、发生地点、上报用户 ID、审核状态

1NF:每个列包含不可再分的原子值,没有重复的列。

2NF:每个列都完全依赖于主键 (EventID),不存在部分依赖。

3NF:每个列都直接依赖于主键 (EventID),不存在传递依赖。

结论:事件表满足 1NF、2NF 和 3NF。

\paragraph{多媒体文件表}

列:文件 ID (主键)、事件 ID、文件路径、文件类型

1NF:每个列包含不可再分的原子值,没有重复的列。

2NF:每个列都完全依赖于主键 (FileID),不存在部分依赖。

3NF:每个列都直接依赖于主键 (FileID),不存在传递依赖。

结论:多媒体文件表满足 1NF、2NF 和 3NF。

\paragraph{通知表}

列:通知 ID (主键)、接收用户 ID、内容、发送时间、状态

1NF:每个列包含不可再分的原子值,没有重复的列。

2NF:每个列都完全依赖于主键 (NotifyID),不存在部分依赖。

3NF:每个列都直接依赖于主键 (NotifyID),不存在传递依赖。

结论:通知表满足 1NF、2NF 和 3NF。

\paragraph{用户事件表}

列:用户 ID (复合主键)、事件 ID (复合主键)

1NF:每个列包含不可再分的原子值,没有重复的列。

2NF:每个列都完全依赖于复合主键 (UserID, EventID),不存在部分依赖。

3NF:每个列都直接依赖于复合主键 (UserID, EventID),不存在传递依赖。

结论:用户事件表满足 1NF、2NF 和 3NF。

通过分析前述设计的每个关系,可以得出它们都满足 1NF、2NF 和 3NF 的要求。每个关系都有合适的主键,并且每个非主属性都完全依赖于主键,且不存在传递依赖。因此,城市犯罪事件管理平台的数据库设计满足第三范式 (3NF)。
